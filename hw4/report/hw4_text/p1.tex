\section{MBRL Problem 1}
% \vspace{-1em}

\subsection{What you will implement}
% \vspace{-0.5em}
Collect a large dataset by executing random actions. Train a neural network dynamics model on this fixed
dataset. The implementation that you will do here will be for training the dynamics model.
\vspace{-0.5em}

\subsection{What code files to fill in}
% \vspace{-0.5em}
The starter code for this assignment can be found at
\vspace{-0.3em}
\begin{itemize}[itemsep=0pt, topsep=0pt, parsep=0pt, partopsep=0pt]
    \item \verb|cs285/agents/model_based_agent.py|: up to (and including) \verb|update_statistics|.
    \item \verb|cs285/scripts/run_hw4.py|: everything except for \verb|collect_mbpo_rollout| at the top of the file.
\end{itemize}
\vspace{-0.5em}

\subsection{What command to run}
% \vspace{-0.5em}
 \begin{lstlisting}[language=bash,breaklines=true]
  python cs285/scripts/run_hw4.py -cfg experiments/mpc/halfcheetah_0_iter.yaml
\end{lstlisting}
\vspace{-0.5em}

This config will only run the first iteration without actually evaluating the policy, meaning it will only train the
ensemble of dynamics models. The code will produce plots inside your logdir that illustrate the full learning
curve of the dynamics models. For the first command, the loss should go below 0.2 by iteration 500.

Modify  \verb|experiments/mpc/halfcheetah_0_iter.yaml| to change some hyperparameters. Try at least
two other configurations of hyperparameters that affect dynamics model training (e.g., number of layers,
hidden size, or learning rate).
\vspace{-0.5em}

\subsection{What to submit:}
% \vspace{-0.5em}
For this question, submit the learning curve for 3 runs total: the initial run with provided
hyperparameters as well as 2 of your own.
Note that for these learning curves, we intend for you to just copy the png images produced by the code.
\vspace{-0.5em}


\begin{figure}[htb]
\centering
\begin{minipage}{0.235\textwidth}
    \centering
    \includegraphics[width=\textwidth]{hw4_text/figures/p1/layers1_h32.png}
    \subcaption{1 Layer}
\end{minipage}%
\hfill
\begin{minipage}{0.235\textwidth}
    \centering
    \includegraphics[width=\textwidth]{hw4_text/figures/p1/layers2_h32.png}
    \subcaption{2 Layers}
\end{minipage}%
\hfill
\begin{minipage}{0.235\textwidth}
    \centering
    \includegraphics[width=\textwidth]{hw4_text/figures/p1/layers3_h32.png}
    \subcaption{3 Layers}
\end{minipage}%
\hfill
\begin{minipage}{0.235\textwidth}
    \centering
    \includegraphics[width=\textwidth]{hw4_text/figures/p1/layers4_h32.png}
    \subcaption{4 Layers}
\end{minipage}
\caption{Effect of Number of Layers (hidden size = 32)}
\vspace{-1em}
\end{figure}
\vspace{-0.5em}

\begin{figure}[htb]
\centering
\resizebox{0.75\textwidth}{!}{%
\begin{minipage}{0.31\textwidth}
    \centering
    \includegraphics[width=\textwidth]{hw4_text/figures/p1/layers1_h32.png}
    \subcaption{Hidden = 32}
\end{minipage}%
\hfill
\begin{minipage}{0.31\textwidth}
    \centering
    \includegraphics[width=\textwidth]{hw4_text/figures/p1/layers1_h64.png}
    \subcaption{Hidden = 64}
\end{minipage}%
\hfill
\begin{minipage}{0.31\textwidth}
    \centering
    \includegraphics[width=\textwidth]{hw4_text/figures/p1/layers1_h128.png}
    \subcaption{Hidden = 128}
\end{minipage}
}
\caption{Effect of Hidden Size (1 layer)}
\vspace{-1.5em}
\end{figure}


\newpage