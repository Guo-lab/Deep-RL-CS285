\section{Submitting the code and experiment runs}
In order to turn in your code and experiment logs, create a folder that contains the following:
\begin{itemize}
    \item A folder named \texttt{data} with all the experiment runs from this assignment.
          % For Q-learning, you need to submit one run for Q1, two runs for Q2, and three runs for Q3. These folders can be copied directly from the \texttt{data} folder. For the actor critic section, likewise submit one folder for each run.
          \textbf{Do not change the names originally assigned to the folders, as specified by  \texttt{exp\_name} in the instructions. Video logging is disabled by default in the code, but if you turned it on for debugging, you will need to run those again with \texttt{--video\_log\_freq -1}, or else the file size will be too large for submission.}
    \item The \texttt{cs285} folder with all the \texttt{.py} files, with the same names and directory structure as the original homework repository (excluding the \texttt{data} folder). Also include any special instructions we need to run in order to produce each of your figures or tables (e.g. ``run python myassignment.py -sec2q1'' to generate the result for Section 2 Question 1) in the form of a README file.
          % Note that this assignment’s plotting must be done in a python script, such that running a single script like this can generate the plot.
\end{itemize}

As an example, the unzipped version of your submission should result in the following file structure. \textbf{Make sure that the submit.zip file is below 15MB and that they include the prefix \texttt{q1\_}, \texttt{q2\_}, \texttt{q3\_}, etc.}

\begin{forest}
    for tree={
    font=\ttfamily,
    grow'=0,
    child anchor=west,
    parent anchor=south,
    anchor=west,
    calign=first,
    edge path={
            \noexpand\path [draw, \forestoption{edge}]
            (!u.south west) +(7.5pt,0) |- node[fill,inner sep=1.25pt] {} (.child anchor)\forestoption{edge label};
        },
    before typesetting nodes={
            if n=1
                {insert before={[,phantom]}}
                {}
        },
    fit=band,
    before computing xy={l=15pt},
    }
    [submit.zip
    [data
    [hw3\_dqn\_...
    [events.out.tfevents.1567529456.e3a096ac8ff4]
    ]
    [hw3\_sac\_...
    [events.out.tfevents.1567529456.e3a096ac8ff4]
    % [checkpoint]
    % [policy\_itr\_0.data-00000-of-00001]
    % [...]
    ]
    [...]
    ]
    [cs285
        [agents
                [soft\_actor\_critic.py]
                [dqn\_agent.py]
        ]
        [...]
    ]
    [README.md]
    [...]
    ]
\end{forest}
% \todo{tree probably needs to be updated}
% \end{question}

If you are a Mac user, \textbf{do not use the default ``Compress'' option to create the zip}. It creates artifacts that the autograder does not like. You may use \texttt{zip -vr submit.zip . -x "*.DS\_Store"} from your terminal from within the top-level \verb|cs285| directory.

Turn in your assignment on Gradescope. Upload the zip file with your code and log files to \textbf{HW3 Code}, and upload the PDF of your report to \textbf{HW3}.
